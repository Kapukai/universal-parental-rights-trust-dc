\documentclass[12pt,letterpaper]{article}
\usepackage[margin=1in]{geometry}
\usepackage{setspace}
\usepackage{hyperref}
\usepackage{datetime}
\usepackage{amsfonts}
\usepackage{parskip}
\usepackage{ulem}

\hypersetup{
  colorlinks=true,
  linkcolor=blue,
  urlcolor=blue,
  pdftitle={Universal Parental Trust Declaration (D.C.)},
  pdfauthor={Kapukai Governance Lab}
}

\title{\textbf{Universal Parental Trust Declaration and Notice of Legal Standing \\ (District of Columbia)}}
\date{\today}
\author{}

\begin{document}
\maketitle
\onehalfspacing

\section*{Purpose and Scope}

This document serves as a sworn parental trust declaration, created under District of Columbia and United States federal law, to assert and protect the constitutional rights of parents and children facing unlawful interference. It is admissible in legal proceedings, functions as both notice and affidavit, and may be used proactively or defensively in court filings or administrative records.

\section*{I. Declarant Identity and Jurisdictional Notice}

I, \textbf{[Full Legal Name]}, born on \textbf{[Date of Birth]}, residing at \textbf{[Address]}, declare under penalty of perjury under the laws of the District of Columbia and the United States that the following is true and correct.

This sworn declaration is made pursuant to:
\begin{itemize}
  \item \href{https://code.dccouncil.gov/us/dc/council/code/sections/1-623.24}{\textbf{D.C. Code § 1–623.24}} – Declarations under penalty of perjury;
  \item \href{https://www.dccourts.gov/sites/default/files/2017-03/Superior-Court-Rules-of-Civil-Procedure.pdf}{\textbf{D.C. Superior Court Rule 9-I(e)}} – Acceptance of sworn declarations without notarization;
  \item \href{https://www.law.cornell.edu/uscode/text/28/1746}{\textbf{28 U.S.C. § 1746}} – Federal provision for sworn declarations without notarization;
  \item \href{https://www.law.cornell.edu/constitution/amendmentxiv}{\textbf{U.S. Constitution, 14th Amendment}} – Equal protection and due process.
\end{itemize}

\section*{II. Establishment of Parental Trust}

As a parent and natural guardian, I hereby establish a living trust over my child(ren) to protect their person, estate, and liberty, free from unlawful interference, fraud, abduction, or state-sponsored trafficking.

\textbf{Child(ren):}
\begin{itemize}
  \item \textbf{[Child’s Full Name]}, born on \textbf{[DOB]}
  \item \textbf{[Additional child, if any]}
\end{itemize}

This trust:
\begin{itemize}
    \item Affirms my lawful parental rights;
    \item Asserts protective custody as a living trustor;
    \item Functions as prima facie notice to any state actor, agency, or entity;
    \item Operates as a standing objection to all attempts to sever, terminate, or falsify the legal relationship between myself and the above-named child(ren).
\end{itemize}

\section*{III. Notice of Fraud, Racketeering, and Due Process Violations}

This trust serves as a sworn declaration and notice that:

\begin{itemize}
  \item Any removal, seizure, or obstruction of the parent-child bond without a finding of parental unfitness by clear and convincing evidence—as required by \href{https://supreme.justia.com/cases/federal/us/455/745/}{\textbf{Santosky v. Kramer, 455 U.S. 745 (1982)}}—is a violation of due process.
  \item Any act by Child Protective Services (CPS), social workers, contractors, or courts to fabricate records, suppress evidence, or use coercive psychological tactics constitutes predicate acts under \href{https://www.law.cornell.edu/uscode/text/18/1961}{\textbf{18 U.S.C. § 1961 (RICO)}}.
  \item Participation in schemes to generate revenue from removal, foster care placements, or adoption incentives (e.g., Title IV-E funds) without lawful cause constitutes systemic fraud and human trafficking under \href{https://www.law.cornell.edu/uscode/text/18/1591}{\textbf{18 U.S.C. § 1591}}.
\end{itemize}

I do not consent to unlawful surveillance, coercive evaluations, psychological manipulation, or child removal under color of law.

\section*{IV. Standing Objection and Demand for Protection}

I issue a standing legal objection to any court order, administrative process, or agency interference that lacks:
\begin{itemize}
  \item Proper jurisdiction;
  \item Proper service;
  \item Constitutional compliance;
  \item Due process in accordance with federal and international law.
\end{itemize}

I assert the right to raise my child(ren) in accordance with my faith, culture, and moral conscience, free from systemic retaliation or data manipulation.

\section*{V. Optional Emergency Witness}

In the event of an emergency, or if I am incapacitated, the following person may be contacted:

\begin{itemize}
  \item \textbf{Name:} \rule{10cm}{0.4pt}
  \item \textbf{Relationship:} \rule{9cm}{0.4pt}
  \item \textbf{Phone/Email:} \rule{9cm}{0.4pt}
\end{itemize}

\section*{VI. Sworn Declaration Under Penalty of Perjury}

Pursuant to D.C. Code § 1–623.24 and 28 U.S.C. § 1746, I affirm under penalty of perjury that the foregoing is true and correct.

\vspace{2em}

\noindent
\textbf{Signed:} \rule{8cm}{0.4pt} \\
\textbf{Date:} \rule{5cm}{0.4pt} \\
\textbf{Executed in:} \rule{6cm}{0.4pt}

\vspace{2em}

\section*{VII. Optional Notary Acknowledgment (If Required)}

Sworn and subscribed before me this \rule{3cm}{0.4pt} day of \rule{5cm}{0.4pt}, 20\_\_. \\
\textbf{Notary Public:} \rule{7cm}{0.4pt} \\
My Commission Expires: \rule{5cm}{0.4pt}

\vspace{2em}

\section*{Legal Notice}

This document is a sworn legal instrument. It may be used by any competent individual to assert their rights under D.C. and federal law. While it may be filed independently, users are advised to consult counsel when engaging in litigation or complex jurisdictional disputes.

\end{document}
